\begin{titlepage}
  \begin{center}

  {\Huge BUS\_BLOCK\_RAM}

  \vspace{25mm}

  \includegraphics[width=0.90\textwidth,height=\textheight,keepaspectratio]{img/AFRL.png}

  \vspace{25mm}

  \today

  \vspace{15mm}

  {\Large Jay Convertino}

  \end{center}
\end{titlepage}

\tableofcontents

\newpage

\section{Usage}

\subsection{Introduction}

\par
Selectable BUS block RAM for any FPGA target. Currently supports wishbone classic or AXI lite. This will create FPGA block RAM that is accessable
via the selected bus.

\subsection{Dependencies}

\par
The following are the dependencies of the cores.

\begin{itemize}
  \item fusesoc 2.X
  \item iverilog (simulation)
  \item cocotb (simulation)
\end{itemize}

\input{src/fusesoc/depend_axi_lite_block_ram.tex}

\subsubsection{wishbone\_classic\_block\_ram Depenecies}
\begin{itemize}
\item dep
	\begin{itemize}
	\item AFRL:utility:helper:1.0.0
	\item AFRL:ram:dc\_block\_ram:1.0.0
	\item AFRL:bus:up\_wishbone\_classic:1.0.0
	\end{itemize}
\item dep\_tb
	\begin{itemize}
	\item AFRL:simulation:clock\_stimulator
	\item AFRL:utility:sim\_helper
	\end{itemize}
\end{itemize}


\subsection{In a Project}
\par
Connect the device using the bus selected, see \ref{Module Documentation} for details

\section{Architecture}
\par
There are two bus block RAM cores. The AXI lite block RAM and the Wishbone Classic block RAM.

AXI lite block RAM is made up of the following modules.
\begin{itemize}
  \item \textbf{up\_axi} Convert AXI lite to the Analog Devices uP BUS. (see core for documentation).
  \item \textbf{dc\_block\_ram} Provides a dual clock block RAM. (see core for documentation).
\end{itemize}

\par
This core has 1 always blocks that are sensitive to the positive clock edge.

\begin{itemize}
\item \textbf{register reqest to the acknoledge} Takes the request and registers to the acknoledge. All reads and writes will produce something.
\end{itemize}

Please see \ref{Module Documentation} for more information.

Wishbone Classic block RAM is made up of the following modules.
\begin{itemize}
  \item \textbf{up\_wishbone\_classic} Convert Wishbone Classic to the Analog Devices uP BUS. (see core for documentation).
  \item \textbf{dc\_block\_ram} Provides a dual clock block RAM. (see core for documentation).
\end{itemize}

\par
This core has 1 always blocks that are sensitive to the positive clock edge.

\begin{itemize}
\item \textbf{register reqest to the acknoledge} Takes the request and registers to the acknoledge. All reads and writes will produce something.
\end{itemize}

Please see \ref{Module Documentation} for more information.

\section{Building}

\par
The BUS block RAM cores are written in Verilog 2001. They should synthesize in any modern FPGA software. The core comes as a fusesoc packaged core and can be included in any other core. Be sure to make sure you have meet the dependencies listed in the previous section. Linting is performed by verible using the lint target.

\subsection{fusesoc}
\par
Fusesoc is a system for building FPGA software without relying on the internal project management of the tool. Avoiding vendor lock in to Vivado or Quartus.
These cores, when included in a project, can be easily integrated and targets created based upon the end developer needs. The core by itself is not a part of
a system and should be integrated into a fusesoc based system. Simulations are setup to use fusesoc and are a part of its targets.

\subsection{Source Files}

\subsubsection{axi\_lite\_block\_ram File List}
\begin{itemize}
\item src
	\begin{itemize}
	\item src/axi\_lite\_block\_ram.v
	\end{itemize}
\item tb
	\begin{itemize}
	\item {'tb/tb\_axi\_lite\_slave.v': {'file\_type': 'verilogSource'}}
	\end{itemize}
\item tb\_cocotb
	\begin{itemize}
	\item {'tb/tb\_axi\_lite\_cocotb.py': {'file\_type': 'user', 'copyto': '.'}}
	\item {'tb/tb\_axi\_lite\_cocotb.v': {'file\_type': 'verilogSource'}}
	\end{itemize}
\end{itemize}


\subsubsection{wishbone\_classic\_block\_ram File List}
\begin{itemize}
\item src
	\begin{itemize}
	\item src/wishbone\_classic\_block\_ram.v
	\end{itemize}
\item tb
	\begin{itemize}
	\item {'tb/tb\_wishbone\_slave.v': {'file\_type': 'verilogSource'}}
	\end{itemize}
\item tb\_cocotb
	\begin{itemize}
	\item {'tb/tb\_wishbone\_cocotb.py': {'file\_type': 'user', 'copyto': '.'}}
	\item {'tb/tb\_wishbone\_cocotb.v': {'file\_type': 'verilogSource'}}
	\end{itemize}
\end{itemize}


\subsection{Targets}

\subsubsection{axi\_lite\_block\_ram Targets}
\begin{itemize}
\item default
	\begin{itemize}
	\item[$\space$] Info: Default for IP intergration.
	\end{itemize}
\item sim
	\begin{itemize}
	\item[$\space$] Info: Simple read/write register check.
	\end{itemize}
\item sim\_cocotb
	\begin{itemize}
	\item[$\space$] Info: Cocotb unit tests
	\end{itemize}
\end{itemize}


\subsubsection{wishbone\_classic\_block\_ram Targets}
\begin{itemize}
\item default
	\begin{itemize}
	\item[$\space$] Info: Default for IP intergration.
	\end{itemize}
\item sim
	\begin{itemize}
	\item[$\space$] Info: Default for IP intergration.
	\end{itemize}
\end{itemize}


\subsection{Directory Guide}

\par
Below highlights important folders from the root of the directory.

\begin{enumerate}
  \item \textbf{docs} Contains all documentation related to this project.
    \begin{itemize}
      \item \textbf{manual} Contains user manual and github page that are generated from the latex sources.
    \end{itemize}
  \item \textbf{src} Contains source files for the core
  \item \textbf{tb} Contains test bench files for iverilog and cocotb
    \begin{itemize}
      \item \textbf{cocotb} testbench files
    \end{itemize}
\end{enumerate}

\newpage

\section{Simulation}
\par
There are a few different simulations that can be run for this core. All currently use iVerilog (icarus) to run. The first is iverilog, which
uses verilog only for the simulations. The other is cocotb. This does a unit test approach to the testing and gives a list of tests that pass
or fail.

\subsection{iverilog}
\par
All simulation targets that do NOT have cocotb in the name use a verilog test bench with verilog stimulus components. For AXI Lite/Wishbone these
are very simple read/writes without data verification.

\subsection{cocotb}
\par
To use the cocotb tests you must install the following python libraries. Only AXI lite is supported for cocotb sims at the moment.
\begin{lstlisting}[language=bash]
  $ pip install cocotb
  $ pip install cocotbext-axi
\end{lstlisting}

Then you must use the cocotb sim target. In this case it is sim\_cocotb. This target can be run with various parameters.
\begin{lstlisting}[language=bash]
  $ fusesoc run --target sim_cocotb AFRL:ram:axi_lite_block_ram:1.0.0 --BUS_WIDTH=8
\end{lstlisting}

\newpage

\section{Module Documentation} \label{Module Documentation}

\par
There are two different BUS block RAM modules that can be used in a project.

\begin{itemize}
\item \textbf{axi\_lite\_block\_ram} AXI lite block RAM\\
\item \textbf{wishbone\_classic\_block\_ram} Wishbone Classic block RAM\\
\item \textbf{tb\_axi\_lite\_cocotb-py} Python axi lite cocotb test bench\\
\item \textbf{tb\_axi\_lite\_cocotb-v}  Verilog axi lite cocotb test bench\\
\item \textbf{tb\_axi\_lite\_slave-v} Verilog test bench for axi lite.\\
\item \textbf{tb\_wishbone\_cocotb-py} Python wishbone cocotb test bench\\
\item \textbf{tb\_wishbone\_cocotb-v}  Verilog wishbone cocotb test bench\\
\item \textbf{tb\_wishbone\_slave-v} Verilog test bench for wishbone.\\
\end{itemize}
The next sections document the module.

